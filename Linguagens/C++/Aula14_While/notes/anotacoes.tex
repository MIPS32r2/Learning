\documentclass[a4paper, 12pt]{article}
\usepackage[top=2cm, bottom=2cm, left=2.5cm, right=2.5cm]{geometry}
\usepackage[utf8]{inputenc}
\usepackage{listings}
\usepackage{color}

\title{Utilização do laço de Repetição While}
\author{Jacaré}
\date{\today}

\begin{document}

\maketitle

\section{Introdução}

O While é usado como estrutura de repetição, enquanto verdadeiro executa uma intrução ou uma série de instruções.
dentro do código podemos ter um laço onde somamos uma varivável que contém valor nulo e podemos adicionar com base na estrutura do while

\section{Utilizações}

\begin{enumerate}
  \item Pode-se fazer uma contagem ao contrário.
\begin{lstlisting}[language=C++, caption={Contagem ao contrário}, basicstyle=\ttfamily\footnotesize, frame=single]
#include <iostream>
using namespace std;

int main(){

  int cont;

  cont=20
  while(cont>0){
    cout << "Jacarino - " << cont << "\n";
    //n++;
    cont--;
  }

  cout << "\nRotina finalizada\n";

  return 0;
}
\end{lstlisting}

No exemplo acima podemos ver que a variável \textbf{const} recebe o valor 20, dentro do while vemos que a condição da estrutura é que o valor da variável seja menor do que 0, tornando a condição falsa inicialmente.

Mas aí vem a pergunta, se a condição é falsa, o programa não fica em loop infinito?
E a resposta é Não! observando o código uma linha em específico é responsável pela encerramento do programa.


\begin{lstlisting}[language=C++, basicstyle=\ttfamily\footnotesize, frame=single]
cont--;
\end{lstlisting}

Acontece que o \textbf{"--"} tem a propriedade de incrementar um número a variável, diferente de quando usamos apenas um traço, assim subtraindo 1 toda vez que o while é executado.

\begin{lstlisting}[language=C++, caption={Incrementar ou decrementar direto no corpo}, basicstyle=\ttfamily\footnotesize, frame=single]
cont=20;
while(cont++>0){
cout << "Jacarino -"  << cont << "\n";
//n++;
}
\end{lstlisting}


\end{enumerate}

\section{Pontos de Atenção}

\begin{enumerate}
  \item A condição de parada deve ser sempre verificada
\end{enumerate}

\begin{lstlisting}[language=C++, caption={Utiizando o Break para finalizar o programa}, basicstyle=\ttfamily\footnotesize, frame=single]
#include <iostream>
using namespace std;

int main(){
  cont=20
  while(cont>0){
    cout << "Jacarino - " << cont << "\n";
      //n++;
      //cont--;
    if(count == 10){
      break;
    } 
  }
return 0;
}
\end{lstlisting}
\end{document}
